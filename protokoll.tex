%
% Laborbericht - Vorlage f�r das Grundlagen der Elektrotechnik, Labor
%                        (c) 2015 IGTE
%---------------------------------------------------------
% Dieses \LaTeX Template kann als Grundlage f�r die Erstellung der
% Laborberichte im Rahmen der Lehrveranstaltung
% "Grundlagen der Elektrotechnik, Labor" verwendet werden.
%----------------------------------------------------------
%
\documentclass[12pt,a4paper,ngerman]{article}
\usepackage{getlab}
\begin{document}
%
%---------------------------------------------------------
\GETHeader                          %  Bitte Ausf�llen!!!
%----------------------------
{�bung 1: Messen Grundlegender Gr�ssen}                       %  �bungstitel
%----------------------------
{22.03.2017}                        %  �bungsdatum
%----------------------------
{338}                            %  Gruppen-Nr.
%----------------------------
{Protokollf�hrer}                   % Name des Protokollf�hrers oder der Protokollf�hrerin
%----------------------------
{
1.~Student\\
2.~Student\\                    %  �bungsteilnehmerInnen
%3.~Student(in)\\                    %  ...bei <4 Teilnehmer auskommentieren
%4.~Student(in)\\
}
%----------------------------
{Laborleiter(in)}                       %  Laborleiter
{Betreuer(in)}                          %  Betreuer
%----------------------------
{Graz}                              %  Ort der Protokollerstellung
{\today}                            %  Datum Protokollerstellung
%----------------------------------------------------------
%----------------------------------------------------------
% Inhaltsverzeichnis
%
% \tableofcontents            % uncomment if you want
%----------------------------------------------------------
\clearpage
\pagestyle{plain}
%
% Los gehts mit (Teil-)�bung 1
%
% Titel der �bung
\section{Messen von Strom und Spannung}
%
\subsection{Aufgabenstellung}


\subsection{Tabellen}
%


\subsection{Formeln}


\subsection{Diagramme}

\clearpage  % damit alle Diagramme vor dem Ger�teverzeichnis platziert werden

\subsection{Ger�teverzeichnis}
%
1\,Stk. Kondensator 150\,nF.\\
1\,Stk. Frequenzgenerator Hameg HM8030-6.\\
2\,Stk. Digitales Multimeter Fluke 79.\\


\subsection{Diskussion}



%\section{�bung~2: \dots}   % Weitere unabh�ngige Teil�bungen
%
%\vspace*{2cm}
% \section*{Unterschriften der �bungsteilnehmer}   % aktivieren bei Protokollabgabe in Papierform
%
\vspace*{2cm}
%
Graz, am \today  % Unterschriften auf Ausdruck!
%
\vspace*{1cm}
%
%-----------------------------------------------
% Bibliographie, wenn notwendig.
% Ansonsten: Auskommentieren!
%
\bibliographystyle{plain}
%\bibliography{getlab}
\end{document}
